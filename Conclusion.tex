
% This work is licensed under the Creative Commons Attribution-Share Alike 2.0 France License.
% To view a copy of this license, visit http://creativecommons.org/licenses/by-sa/2.0/fr/legalcode
% or send a letter to Creative Commons, 171 Second Street, Suite 300, San Francisco, California, 94105, USA.


\chapter{Où aller à partir de ce point?}

Congratulations! Vous avez été jusqu'au bout. Ce que vous avez probablement appris de ce livre sont les concepts fondamentaux qui rendront l'apprentissage d'autres langages de programmation beaucoup plus simple. Même si Python est un langage de programmation brillant, un langage particulier n'est pas toujours le meilleur outil pour toutes les tâches. Donc n'ayez pas peur de regarder pour d'autres manières de programmer votre ordinateur si cela vous intéresse.

Par exemple, si vous êtes intéressés dans la programmation de jeux, vous pouvez peut-être regarder quelque chose comme BlitzBasic (\url{www.blitzbasic.com}) qui utilise le langage de programmation Basic. Ou encore, Flash qui est utilisé dans de nombreux sites web pour des animations et des jeux; par exemple le site de Nickelodeon (\url{www.nick.com}) utilise  beaucoup de Flash. Si vous êtes intéressez par la programmation en Flash un bon départ sera sûrement «~Flash CS4 pour les nuls~» un livre qui existe en version poche; vous pouvez aussi lire «~ActionScript 3 - Développez des jeux en Flash~». Rechercher «~jeu Flash~» sur \url{http://www.eyrolles.com} ou sur \url{www.amazon.fr}. Certains autres livres comme «~Hands on Darkbasic Pro: A Self-study Guide to Games Programming: v. 2~» ou «~Game Programming for Teens~» n'existent qu'en anglais.
Soyez conscients que les outils de développement BlitzBasic, DarkBasic et Flash coûtent de l'argent et que les livres en parlant sont généralement payants à l'inverse de Python. L'implication de vos parents doit donc être acquise avant même de pouvoir commencer.

NDT: Les langages les plus utilisés pour programmer sont le C et le C++ pour les jeux sur ordinateurs et consoles et Java (J2ME) pour les téléphones cellulaires. Ces langages ont l'avantage d'être gratuits et de disposer de nombreuses documentations. Par contre ils sont moins simples d'abord que le Basic. Pour les sites web le JavaScript est maintenant utilisé pour des applications complexes.

Si vous voulez continuer avec Python pour faire des jeux vous devriez aller sur \url{www.pygame.org} où vous trouverez la bibliothèque la plus utilisée pour faire des jeux. Deux livres en anglais pourront alors vous être utiles : «~Beginning Game Development with Python and Pygame: From Novice to Professional~» et «~Game Programming With Python~».


Si vous n'êtes pas spécialement intéressés dans la programmation de jeux mais que vous voulez apprendre plus à propos de Python vous pouvez commencer par «~Apprendre à programmer avec Python~» \url{http://www.framasoft.net/IMG/pdf/python_notes-2.pdf}. Vous pouvez aussi jeter un coup d'œil à «~Dive into Python~» de Mark Pilgrim \url{www.diveintopython.org}. Il y a aussi un tutoriel libre disponible sur \url{http://docs.python.org/tut/tut.html}.

Il y a tout un ensemble de sujets que nous n'avons pas couvert dans cette simple introduction donc, tout du moins pour Python, il y a encore beaucoup à apprendre et avec quoi s'amuser. Bonne chance et amusez vous à programmer.
 
